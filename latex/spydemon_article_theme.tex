\documentclass[a4paper, 10pt]{article}

\usepackage[utf8]{inputenc}													% Encodage correct pour le français. 
\usepackage[T1]{fontenc}      												% « Coupage » de mot compatible avec les mots accentués.
\usepackage[francais]{babel} 													% Configuration de LaTeX pour utiliser les guillemets & cie français. 
\usepackage{layout}																% Customisation des marges. 
\usepackage[top=2cm, bottom=2cm, left=2cm, right=2cm]{geometry}	% Réglage des marges.
\usepackage{charter}																% Pack de polices à utiliser.
\usepackage{url}																	% Rendre les liens cliquables.
\usepackage{fancyhdr}															% Customisation des en-tête et pied de pages.
\usepackage{multicol}															% Possibilité d'utiliser des colonnes.
\usepackage{titlesec} 															% Modification du style des sections.
\usepackage{lastpage}															% Pour récupérer le numéro de la derière page pour l'afficher dans le footer.
\usepackage{MnSymbol,wasysym}													% Pour avoir accès à des carractères cools.

\makeatletter

% Customisation du pied de pages.
\pagestyle{fancyplain}
\fancyhf{}																			% On désactive les en-tête et pied de page par défaut.
\lfoot{ \sc \@footer}	 														% Le texte à écrire en pied de page gauche. 
\rfoot{\thepage \slash \pageref{LastPage}}								% Le texte à écrire en pied de page droit.
\renewcommand{\headrulewidth}{0pt}											% On enlève la ligne dessinée par défaut dans l'en-tête de la page.
\setcounter{secnumdepth}{0} 													% On désactive la numérotation des sections.

% Customisation de la partie titre du documement. 
% Environnement titlepage à utiliser pour créer une page dédiée.
\def\maketitle{
	\begin{center}
		{\sc \Huge \textbf{\@title}} 
		\vskip 0.2cm
		{\textit {\@author}} \\
		{\texttt{\@email}} \\
		{\@date}
		\vskip 1cm
	\end{center}
}

%Définition de variables utiles.
\def\email#1{\def\@email{#1}}
\def\footer#1{\def\@footer{#1}}

% Customisation de l'affichage des sections et subsections.
\def\section{\@startsection {section}{1}{\z@}{-3.5ex plus -1ex minus 
    -.2ex}{2.3ex plus .2ex}{\centering\Large\bfseries\scshape}}
\def\subsection{\@startsection{subsection}{2}{\z@}{-3.25ex plus -1ex minus 
   -.2ex}{1.5ex plus .2ex}{\centering\scshape}}

% Bloc d'introduction.
\newenvironment*{introduction}{%
	\begin{center} \begin{minipage}{15cm} \parindent=15pt}
	{\end{minipage} \vskip 0.5cm \rule{150pt}{.5pt} \end{center}}

% Bloc de conclusion.
\newenvironment*{conclusion}{%
	\vskip 0.5cm \begin{center} \begin{minipage}{15cm} \parindent=15pt}
	{\end{minipage} \end{center}}

% Mise en place d'une ligne entre les colonnes.
%\setlength\columnseprule{0.2pt} 

\makeatother
